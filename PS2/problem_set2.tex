\documentclass[12pt]{article}

\usepackage{amsmath}
\usepackage[margin=2.5cm]{geometry}
\usepackage{csc}
\usepackage{textcomp}

% Document metadata
\title{CSC165H1 Problem Set 2}
\author{Yulin WANG, Qidi Zhou, Dana Zhao}
\date{Wednesday February 7}
% Document starts here
\begin{document}
\maketitle

\section*{1. AND vs. IMPLIES} 
\vspace{20pt}
\vspace{20pt}

\subsection*{a)}
Proof: \\
Header:

Let n$\in$ $\mathbb{N}$

Assume that n $\textgreater$ 15

We want to prove that $n^3$ - 10$n^2$ + 3 \ $\geq$\ 165. \\
~\\
Body:
\begin{align*}
n^3-10n^2+3 &\geq n^3 - 10n^2  \quad(since \ 3 \geq 0) \\
&= n^2(n-10) \\
&\geq  n^2 \times 1 = n^2 = 15^2 \quad (since \ n > 15  \Rightarrow n - 10 \geq 1) \\
&\geq 165
\end{align*}\\
Therefore, we have proven that $\forall$n $\in$ $\mathbb{N}$, n $\textgreater$ 15 $\Rightarrow$ n$^3$ -10n$^2$ + 3\ $\geq$ 165. \\

$\hfill\square$ 
\newpage
\subsection*{b)}
Since this statement is False, so we want to disprove it.\\
We can prove that its negation is True.\\
Negation of the original statement: $\exists$n $\in$ $\mathbb{N}$, n $\leq$ 15 $\vee$ $n^3$ - 10$n^2$ + 3\ $\textless$ 165.\\
Proof:\\
Header:

Let n = 1, then n $\in$ $\mathbb{N}$

We want to prove  n $\leq$ 15 $\vee$ n$^3$ - 10n$^2$ + 3\ $\textless$\ 165.\\
Body:

Since n = 1, so n $\leq$ 15.

Thus, n $\leq$ 15 $\vee$ n$^3$ - 10n$^2$ + 3\ $\textless$\ 165 is True. \\
~\\
Therefore, negation of the original statement is True, then the original statement is False.

$\hfill\square$ 

\vspace{20pt}

\section*{2. Ceiling function}
\subsection*{a)}
Translation: $\forall$n,m $\in$ $\mathbb{N}$, n $\textless$ m $\Rightarrow$ $\lceil$$\frac{m-1}{m}$$\cdot$ n$\rceil$ = n \\
Proof:\\
Header:

Let n, m $\in$ $\mathbb{N}$

Assume that n $\textless$ m

We want to prove that $\lceil$$\frac{(m-1)}{m}$$\cdot$ n$\rceil$ = n. \\
Body:
\begin{align*}
\lceil \frac{(m-1)}{m} \cdot n \rceil  &= \lceil (1 - \frac{1}{m})\cdot n\rceil  \\
&=\lceil - \frac{n}{m} + n\rceil \\
\end{align*}

By Fact 2, since -$\frac{n}{m}$ $\in$ $\mathbb{R}$ and n$\in$ $\mathbb{Z}$, so we have:

$\quad$ $\quad$ $\quad$ $\quad$ $\quad$ $\lceil$-$\frac{n}{m}$ + n$\rceil$ = $\lceil$-$\frac{n}{m}$$\rceil$ + n

Since n, m $\in$ $\mathbb{N}$ and n $\textless$ m, we know that 0 $\leq$ $\frac{n}{m}$ $\textless$ 1

Then, -1 $\textless$ - $\frac{n}{m}$ $\leq$ 0

So, $\lceil$-$\frac{n}{m}$$\rceil$ = 0

Thus, $\lceil$ $\frac{(m-1)}{m}$ $\cdot$ n $\rceil$ = $\lceil$-$\frac{n}{m}$$\rceil$ + n = 0 + n = n\\
~\\
Therefore, we have proven that $\forall$ n,m $\in$ $\mathbb{N}$, n $\textless$ m $\Rightarrow$ $\lceil$$\frac{(m-1)}{m}$$\cdot$n$\rceil$ = n.

$\hfill\square$

\newpage

\subsection*{b)}
\vspace{20pt}
Translation: \\
$\forall$n $\in$ $\mathbb{N}$, (50$\mid$nextFifty(n) $\wedge$ nextFifty(n) $\geq$ n) $\wedge$ ($\forall$k $\in$ $\mathbb{N}$, 50\textbar k $\wedge$ k $\geq$ n $\Rightarrow$ k $\geq$ nextFifty(n)) \\
~\\
$\Leftrightarrow$ $\forall$n $\in$ $\mathbb{N}$, (50$\mid$50$\cdot$$\lceil$$\frac{n}{50}$$\rceil$ $\wedge$ 50$\cdot$$\lceil$$\frac{n}{50}$$\rceil$ $\geq$ n) $\wedge$ ($\forall$k $\in$ $\mathbb{N}$, 50\textbar k $\wedge$ k $\geq$ n $\Rightarrow$ k $\geq$ 50$\cdot$$\lceil$$\frac{n}{50}$$\rceil$) \\
~\\
Proof: \\
Header:

Let n$\in$ $\mathbb{N}$\\
Body:\\
~\\
Firstly, we want to prove that 50\textbar50$\cdot$$\lceil$$\frac{n}{50}$$\rceil$ $\wedge$ 50$\cdot$$\lceil$$\frac{n}{50}$$\rceil$ $\geq$ n

Since 1\textbar$\lceil$$\frac{n}{50}$$\rceil$, so 50\textbar50$\cdot$$\lceil$$\frac{n}{50}$$\rceil$ $\quad$ (multiply 50 on each side)

Since $\lceil$x$\rceil$ $\geq$ x, so $\lceil$$\frac{n}{50}$$\rceil$ $\geq$ $\frac{n}{50}$

Thus 50$\cdot$$\lceil$$\frac{n}{50}$$\rceil$ $\geq$ n (multiply 50 on each side) \\
~\\
Therefore, we have proven that 50\textbar50$\cdot$$\lceil$$\frac{n}{50}$$\rceil$ $\wedge$ 50$\cdot$$\lceil$$\frac{n}{50}$$\rceil$ $\geq$ n \\
~\\
Secondly, we want to prove that $\forall$k $\in$ $\mathbb{N}$, 50$\mid$k $\wedge$ k $\geq$ n $\Rightarrow$ k$\geq$ 50$\cdot$$\lceil$$\frac{n}{50}$$\rceil$ 

Let k$\in$ $\mathbb{N}$

Assume that 50$\mid$k $\wedge$ k $\geq$ n

Since 50$\mid$k represents $\exists$m$\in$ $\mathbb{Z}$, k = 50m $\geq$ n

So m $\geq$ $\frac{n}{50}$ (divide 50 on each side)

Then by Fact 1, we know m$\geq$ $\lceil$$\frac{n}{50}$$\rceil$,

so 50m $\geq$ 50$\cdot$$\lceil$$\frac{n}{50}$$\rceil$, then k$\geq$ 50$\cdot$$\lceil$$\frac{n}{50}$$\rceil$ \\
~\\
Therefore, we have proven that $\forall$k $\in$ $\mathbb{N}$, 50$\mid$k $\wedge$ k $\geq$ n $\Rightarrow$ k$\geq$ 50$\cdot$$\lceil$$\frac{n}{50}$$\rceil$ \\
~\\
In conclusion, we have proven that:\\
$\forall$n $\in$ $\mathbb{N}$, (50$\mid$nextFifty(n) $\wedge$ nextFifty(n) $\geq$ n) $\wedge$ ($\forall$k $\in$ $\mathbb{N}$, 50\textbar k $\wedge$ k $\geq$ n $\Rightarrow$ k $\geq$ nextFifty(n)) \\

$\hfill\square$

\newpage



\section*{3. Divisibility}
\subsection*{a)}

Translation: \\
$\forall$n $\in$ $\mathbb{N}$, n $\leq$ 2300 $\Rightarrow$ (49$\mid$n $\Leftrightarrow$  50$\cdot$(50$\cdot$$\lceil$$\frac{n}{50}$$\rceil$ - n) = 50$\cdot$$\lceil$$\frac{n}{50}$$\rceil$) \\
~\\
$\Leftrightarrow$ $\forall$n $\in$ $\mathbb{N}$, n $\leq$ 2300 $\Rightarrow$ (49$\mid$n $\Leftrightarrow$  50$\cdot$$\lceil$$\frac{n}{50}$$\rceil$ - n = $\lceil$$\frac{n}{50}$$\rceil$) \\
~\\
Proof: \\
(1) Firstly, we want to prove the $\Rightarrow$ direction of the statement. \\
Header: 

Let n$\in$ $\mathbb{N}$

Assume that n $\leq$ 2300

We want to prove that 49$\mid$n $\Rightarrow$  50$\cdot$$\lceil$$\frac{n}{50}$$\rceil$ - n = $\lceil$$\frac{n}{50}$$\rceil$ \\
Body:

Assume that 49$\mid$n, which means:

$\exists$k $\in$ $\mathbb{Z}$, n = 49k, so we have:\\
\begin{align*}
50\cdot \lceil \frac{n}{50} \rceil - n &= 50\cdot \lceil \frac{49k}{50} \rceil - 49k \\
&= 50\cdot \lceil \frac{50 - 1}{50} \cdot k\rceil - 49k \\
&= 50\cdot k - 49k \ (since\ n \leq 2300, \ then\  k = \frac{n}{49} \leq \frac{2300}{49} \ \textless \ 50, \ and \ k\in \  \mathbb{N}, \ so\ by\ 2(a)) \\
&= k \\
and\quad \lceil \frac{n}{50} \rceil &= \lceil \frac{49k}{50} \rceil \\
&= \lceil \frac{50 - 1}{50} \cdot k\rceil \\
&= k
\end{align*}

Thus, 50$\cdot$$\lceil$$\frac{n}{50}$$\rceil$ - n = $\lceil$$\frac{n}{50}$$\rceil$ \\
~\\
Therefore, we have proven that: $\forall$n $\in$ $\mathbb{N}$, n $\leq$ 2300 $\Rightarrow$ (49$\mid$n $\Rightarrow$  50$\cdot$(50$\cdot$$\lceil$$\frac{n}{50}$$\rceil$ - n) = 50$\cdot$$\lceil$$\frac{n}{50}$$\rceil$) \\
~\\
(2) Secondly, we want to prove the $\Leftarrow$ direction of the statement. \\
Header: 

Let n$\in$ $\mathbb{N}$

Assume that n $\leq$ 2300

We want to prove that 50$\cdot$$\lceil$$\frac{n}{50}$$\rceil$ - n = $\lceil$$\frac{n}{50}$$\rceil$ $\Rightarrow$ 49$\mid$n \\
Body:

Assume that 50$\cdot$$\lceil$$\frac{n}{50}$$\rceil$ - n = $\lceil$$\frac{n}{50}$$\rceil$

Then, 49$\cdot$$\lceil$$\frac{n}{50}$$\rceil$ = n

So, $\lceil$$\frac{n}{50}$$\rceil$ = $\frac{n}{49}$ (divide 49 on each side)

Since $\lceil$$\frac{n}{50}$$\rceil$$\in$ $\mathbb{Z}$, so $\frac{n}{49}$$\in$ $\mathbb{Z}$, and then 49$\mid$n \\ 
~\\
Therefore, we have proven that: $\forall$n $\in$ $\mathbb{N}$, n $\leq$ 2300 $\Rightarrow$ (50$\cdot$$\lceil$$\frac{n}{50}$$\rceil$ - n = $\lceil$$\frac{n}{50}$$\rceil$ $\Rightarrow$ 49$\mid$n) \\
~\\
In conclusion, we have proven that $\forall$n $\in$ $\mathbb{N}$, n $\leq$ 2300 $\Rightarrow$ (49$\mid$n $\Leftrightarrow$  50$\cdot$(50$\cdot$$\lceil$$\frac{n}{50}$$\rceil$ - n) = 50$\cdot$$\lceil$$\frac{n}{50}$$\rceil$) \\

$\hfill\square$

\vspace{20pt}
\subsection*{b)}
In order to disprove this statement, we can prove its negation is True. \\
Since from 3(a) we know that: \\
$\forall$n $\in$ $\mathbb{N}$, 50$\cdot$(nextFifty(n) - n) = nextFifty(n)) $\Rightarrow$ 49$\mid$n \\
So, we just need to disprove the $\Rightarrow$ direction of this statement. \\
Negation of the $\Rightarrow$ direction of the original statement:

$\exists$n $\in$ $\mathbb{N}$, 49$\mid$n $\wedge$ (50$\cdot$$\lceil$$\frac{n}{50}$$\rceil$ - n $\neq$ $\lceil$$\frac{n}{50}$$\rceil$) \\
Proof:\\
Header:

Let n = 49 $\times$ 50 = 2450, then 49$\mid$n \\
Body: 

50$\cdot$$\lceil$$\frac{n}{50}$$\rceil$ - n = 50$\cdot$$\lceil$$\frac{2450}{50}$$\rceil$ - 2450 = 0

$\lceil$$\frac{n}{50}$$\rceil$ = $\lceil$$\frac{2450}{50}$$\rceil$ = 49

So, 50$\cdot$$\lceil$$\frac{n}{50}$$\rceil$ - n $\neq$ $\lceil$$\frac{n}{50}$$\rceil$ \\
~\\
Thus, $\exists$n $\in$ $\mathbb{N}$, 49$\mid$n $\wedge$ (50$\cdot$$\lceil$$\frac{n}{50}$$\rceil$ - n $\neq$ $\lceil$$\frac{n}{50}$$\rceil$) \\
~\\
Therefore, negation of the original statement is True, then the original statement is False.

$\hfill\square$

\newpage
\section*{4. Functions}
\vspace{20pt}

\subsection*{a)}
$\exists$k $\in$ $\mathbb{R}$, $\forall$x $\in$ $\mathbb{N}$, f(x) $\leq$ k

\vspace{20pt}
\subsection*{b)}
Translation:\\
\(\forall f_1, f_2 : \mathbb{N} \to \mathbb{R}^{\geq 0}\), ($\exists$$k_1$ $\in$ $\mathbb{R}$, $\forall$x $\in$ $\mathbb{N}$, $f_1(x)$ $\leq$ $k_1$) $\wedge$ ($\exists$$k_2$ $\in$ $\mathbb{R}$, $\forall$x $\in$ $\mathbb{N}$, $f_2(x)$ $\leq$ $k_2$)

$\qquad$ $\qquad$ $\qquad$ $\Rightarrow$ ($\exists$$k_3$ $\in$ $\mathbb{R}$, $\forall$x $\in$ $\mathbb{N}$, ($f_1$ + $f_2$)(x) $\leq$ $k_3$) \\
Proof:\\
Header:

Let \(f_1, f_2 : \mathbb{N} \to \mathbb{R}^{\geq 0}\) 

Assume that ($\exists$$k_1$ $\in$ $\mathbb{R}$, $\forall$x $\in$ $\mathbb{N}$, $f_1(x)$ $\leq$ $k_1$) $\wedge$ ($\exists$$k_2$ $\in$ $\mathbb{R}$, $\forall$x $\in$ $\mathbb{N}$, $f_2(x)$ $\leq$ $k_2$)

We want to prove that $\exists$$k_3$ $\in$ $\mathbb{R}$, $\forall$x $\in$ $\mathbb{N}$, ($f_1$ + $f_2$)(x) $\leq$ $k_3$  \\
Body:

Let x $\in$ $\mathbb{N}$

Let $k_3$ = $k_1$ + $k_2$

Since 0 $\leq$ $f_1(x)$ $\leq$ $k_1$ and 0 $\leq$ $f_2(x)$ $\leq$ $k_2$

So, 0 $\leq$ $f_1(x)$ + $f_2(x)$ $\leq$ $k_1$ + $k_2$ = $k_3$

Thus, ($f_1$ + $f_2$)(x) $\leq$ $k_3$ \\
~\\
Therefore, we have proven that: \\
\(\forall f_1, f_2 : \mathbb{N} \to \mathbb{R}^{\geq 0}\), ($\exists$$k_1$ $\in$ $\mathbb{R}$, $\forall$x $\in$ $\mathbb{N}$, $f_1(x)$ $\leq$ $k_1$) $\wedge$ ($\exists$$k_2$ $\in$ $\mathbb{R}$, $\forall$x $\in$ $\mathbb{N}$, $f_2(x)$ $\leq$ $k_2$)

$\qquad$ $\qquad$ $\quad$ $\Rightarrow$ ($\exists$$k_3$ $\in$ $\mathbb{R}$, $\forall$x $\in$ $\mathbb{N}$, ($f_1$ + $f_2$)(x) $\leq$ $k_3$) \\

$\hfill\square$

\newpage

\vspace{20pt}
\subsection*{c)}
\vspace{20pt}
Translation:\\
\(\forall f_1, f_2 : \mathbb{N} \to \mathbb{R}^{\geq 0}\), ($\exists$$k_3$ $\in$ $\mathbb{R}$, $\forall$x $\in$ $\mathbb{N}$, ($f_1$ + $f_2$)(x) $\leq$ $k_3$) 

$\quad$ $\qquad$ $\qquad$ $\Rightarrow$  ($\exists$$k_1$ $\in$ $\mathbb{R}$, $\forall$x $\in$ $\mathbb{N}$, $f_1(x)$ $\leq$ $k_1$) $\wedge$ ($\exists$$k_2$ $\in$ $\mathbb{R}$, $\forall$x $\in$ $\mathbb{N}$, $f_2(x)$ $\leq$ $k_2$) \\
Proof:\\
Header:

Let \(f_1, f_2 : \mathbb{N} \to \mathbb{R}^{\geq 0}\) 

Assume that $\exists$$k_3$ $\in$ $\mathbb{R}$, $\forall$x $\in$ $\mathbb{N}$, ($f_1$ + $f_2$)(x) $\leq$ $k_3$

We want to prove that ($\exists$$k_1$ $\in$ $\mathbb{R}$, $\forall$x $\in$ $\mathbb{N}$, $f_1(x)$ $\leq$ $k_1$) $\wedge$ ($\exists$$k_2$ $\in$ $\mathbb{R}$, $\forall$x $\in$ $\mathbb{N}$, $f_2(x)$ $\leq$ $k_2$) \\
Boby:

Let x $\in$ $\mathbb{N}$

Let $k_1$ = $\frac{k_3}{2}$ - 1

Let $k_2$ = $\frac{k_3}{2}$ + 1

Since, ($f_1$ + $f_2$)(x) $\leq$ $k_3$

So, $f_1(x)$ + $f_2(x)$ $\leq$ $k_3$

Then, $f_1(x)$ $\leq$ $k_3$ - $f_2(x)$ $\leq$ $k_3$ - $k_2$ = $k_3$ - ($\frac{k_3}{2}$ + 1) = $\frac{k_3}{2}$ - 1 = $k_1$

Thus, $f_1(x)$ $\leq$ $k_1$

And, $f_2(x)$ $\leq$ $k_3$ - $f_1(x)$ $\leq$ $k_3$ - $k_1$ = $k_3$ - ($\frac{k_3}{2}$ - 1) = $\frac{k_3}{2}$ + 1 = $k_2$

Thus, $f_2(x)$ $\leq$ $k_2$\\
~\\
Therefore, we have proven that:\\
\(\forall f_1, f_2 : \mathbb{N} \to \mathbb{R}^{\geq 0}\), ($\exists$$k_3$ $\in$ $\mathbb{R}$, $\forall$x $\in$ $\mathbb{N}$, ($f_1$ + $f_2$)(x) $\leq$ $k_3$) 

$\quad$ $\qquad$ $\qquad$ $\Rightarrow$  ($\exists$$k_1$ $\in$ $\mathbb{R}$, $\forall$x $\in$ $\mathbb{N}$, $f_1(x)$ $\leq$ $k_1$) $\wedge$ ($\exists$$k_2$ $\in$ $\mathbb{R}$, $\forall$x $\in$ $\mathbb{N}$, $f_2(x)$ $\leq$ $k_2$) \\

$\hfill\square$






\vspace{20pt}



\end{document}
