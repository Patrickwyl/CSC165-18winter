\documentclass[12pt]{article}

\usepackage{amsmath}
\usepackage[margin=2.5cm]{geometry}
\usepackage{csc}
\usepackage{textcomp}

% Document metadata
\title{CSC165H1 Problem Set 3}
\author{Yulin WANG, Qidi Zhou, Dana Zhao}
\date{Wednesday March 14}
% Document starts here
\begin{document}
\maketitle

\newpage
\section*{1. Special numbers} 
\vspace{20pt}
\vspace{20pt}

Proof by induction:

Define the predicate P(n) as $F_{n}$ - 2 = $\prod_{i=0}^{n-1}$$F_{i}$,

where n is a natural number.

We want to prove that for all n $\in$ $\mathbb{N}$ that P(n) holds.\\
Base Case:

n = 0. We want to prove P(0) is True.

We know that $F_{0}$ - 2 = $2^{1}$ + 1 - 2 = 1 

We also know $\prod_{i=0}^{-1}$$F_{i}$ = 1 (By the definition of empty products in product notation.)

Since left hand side $F_{0}$ - 2 is equal to right hand side $\prod_{i=0}^{-1}$$F_{i}$, then P(0) holds.\\
Induction Step:

Let k $\in$ $\mathbb{N}$, and assume P(k) is True that is  $F_{k}$ - 2 = $\prod_{i=0}^{k-1}$$F_{i}$ is True.

Now we want to show that P(k+1) holds that is  $F_{k+1}$ - 2 = $\prod_{i=0}^{k}$$F_{i}$

For right hand side,  $\prod_{i=0}^{k}$$F_{i}$:
\begin{align*}
 \prod_{i=0}^{k}F_{i} &= F_{0} \cdot F_{1} \cdot F_{2} \dotsb F_{k-1}\cdot F_{k}\\
&= \prod_{i=0}^{k-1}F_{i}  \cdot F_{k}\\
&= (F_{k} - 2) \cdot F_{k} \quad(By\ induction\ hypothesis)\\
&= F_{k}^{2} - 2F_{k}\\
&= (2^{2^{k}}+1)^{2} - 2(2^{2^{k}}+1)\quad(From \ definition \ of \ F_{n})\\
&= (2^{2^{k}})^{2} + 2 \cdot 2^{2^{k}} + 1-2 \cdot 2^{2^{k}}-2\\
&= 2^{2^{k+1}}-1\quad(From \ hint)
\end{align*}

For left hand side,  $F_{k+1}$ - 2:
\begin{align*}
F_{k+1} - 2 &= 2^{2^{k+1}}+1-2\quad(From \ definition \ of \ F_{n})\\
&= 2^{2^{k+1}}-1
\end{align*}
Therefore, left hand side is equal to right hand side, we have proven $\forall$k $\in$ $\mathbb{N}$, P(k+1) holds.\
Then we have proven $\forall$n $\in$ $\mathbb{N}$, $F_{n}$ - 2 = $\prod_{i=0}^{n-1}$$F_{i}$ by induction.

$\hfill\square$ 
\newpage

\vspace{20pt}

\section*{2. Sequences}
\subsection*{a)}
$a_{0}$ = 1 \quad  $a_{1}$ = $\frac{1}{2}$ \quad $a_{1}$ =  $\frac{1}{3}$  \quad $a_{1}$ = $\frac{1}{4}$


\subsection*{b)}
\vspace{20pt}
$\forall$n $\in$ $\mathbb{N}$, $a_{n}$ = $\frac{1}{n + 1}$ \\
~\\
Proof by induction: 

Define the predicate P(n) as $a_{n}$ = $\frac{1}{n + 1}$, where n is a natural number.

We want to prove that for all n $\in$ $\mathbb{N}$ that P(n) holds. \\
Base Case:

n = 0. We want to prove that P(0) is true.

Since $a_{0}$ = $\frac{1}{0 + 1}$ = 1, so P(0) holds. \\
Induction Step:

Let k $\in$ $\mathbb{N}$, and assume that P(k) is true.

That is, we assume that $a_{k}$ = $\frac{1}{k+1}$.

We want to show that P(k+1) holds, that is $a_{k+1}$ = $\frac{1}{k + 2}$.

Since we know:
\begin{align*}
a_{k+1} &= \frac{1}{\frac{1}{a_{k}}+1} \\
&= \frac{1}{(k+1)+1}  \quad (By \ induction \ hypothesis) \\
&= \frac{1}{k+2}
\end{align*}

Thus, P(k+1) holds.\\
Therefore, we have proven that $\forall$n $\in$ $\mathbb{N}$, $a_{n}$ = $\frac{1}{n + 1}$ by induction. \\

$\hfill\square$


\subsection*{c)}

$a_{2,0}$ = 2 \quad $a_{2,1}$ =  $\frac{4}{3}$ \quad $a_{2,2}$ =  $\frac{8}{7}$ \quad $a_{2,3}$ =  $\frac{16}{15}$ \quad \\
~\\
$a_{3,0}$ = 3 \quad $a_{3,1}$ =  $\frac{9}{4}$ \quad $a_{3,2}$ =  $\frac{27}{13}$ \quad $a_{3,3}$ =  $\frac{81}{40}$ \quad

\newpage

\subsection*{d)}

$a_{k,n}$ = $\frac{k^{n+1}}{\frac{k^{n+1}-1}{k-1}}$ = $\frac{k^{n+1}(k-1)}{k^{n+1} - 1}$ \\
~\\
We want to show that $\forall$n $\in$ $\mathbb{N}$, $\forall$k $\in$ $\mathbb{N}$, k $\textgreater$ 1 $\Rightarrow$ $a_{k,n}$ = $\frac{k^{n+1}(k-1)}{k^{n+1} - 1}$ \\
~\\
Proof by induction:

Define the Predicate P(n) as $\forall$k $\in$ $\mathbb{N}$, k $\textgreater$ 1 $\Rightarrow$ $a_{k,n}$ = $\frac{k^{n+1}(k-1)}{k^{n+1} - 1}$,

where n is a natural number.

We want to prove that for all n $\in$ $\mathbb{N}$ that P(n) holds.\\
Base Case:

n = 0. We want to prove that P(0) is true.

Let k $\in$ $\mathbb{N}$, and assume that k $\textgreater$ 1.

Since $a_{k,0}$ =  $\frac{k(k-1)}{k-1}$ = k, so P(0) holds. \\
Induction Step:

Let m $\in$ $\mathbb{N}$, and assume that P(m) is true.

That is, we assume that $\forall$k $\in$ $\mathbb{N}$, k $\textgreater$ 1 $\Rightarrow$ $a_{k,m}$ = $\frac{k^{m+1}(k-1)}{k^{m+1} - 1}$

We want to show that P(m+1) holds.

That is $\forall$k $\in$ $\mathbb{N}$, k $\textgreater$ 1 $\Rightarrow$ $a_{k,m+1}$ = $\frac{k^{m+2}(k-1)}{k^{m+2} - 1}$

Let k $\in$ $\mathbb{N}$, and asuume that k $\textgreater$ 1.

Since we know:
\begin{align*}
a_{k,m+1} &= \frac{k}{\frac{1}{a_{k,m}}+1} \\
&= \frac{k}{\frac{k^{m+1}-1}{k^{m+1}(k-1)}+1} \quad (By \ induction \ hypothesis) \\
&= \frac{k}{\frac{(k^{m+1}-1)+k^{m+1}(k-1)}{k^{m+1}(k-1)}} \\
&= \frac{k\cdot k^{m+1}(k-1)}{(k^{m+1}-1)+k^{m+1}(k-1)} \\
&= \frac{k^{m+2}(k-1)}{k^{m+1}(1+k-1)-1} \\
&= \frac{k^{m+2}(k-1)}{k^{m+1}\cdot k-1} \\
&= \frac{k^{m+2}(k-1)}{k^{m+2} - 1}
\end{align*}

Thus, $a_{k,m+1}$ = $\frac{k^{m+2}(k-1)}{k^{m+2} - 1}$, then P(m+1) is true.\\
Therefore, we have proven that $\forall$n $\in$ $\mathbb{N}$, $\forall$k $\in$ $\mathbb{N}$, k $\textgreater$ 1 $\Rightarrow$ $a_{k,n}$ = $\frac{k^{n+1}(k-1)}{k^{n+1} - 1}$ by induction.\\

$\hfill\square$

\newpage

\section*{3. Properties of Asymptotic Notation}
\vspace{20pt}
\subsection*{a)}

Proof:

Let $f: \mathbb{N} \rightarrow \mathbb{R}^{\geq 0}$

Assume that $f \in \cO (n)$

By definition of big-oh, that is
\[ \exists c, n_{o} \in \mathbb{R}^{+}, \forall n \in \mathbb{N}, n \geq n_{o} \IMP f(n) \leq cn \]

We want to show that $Sum_{f}(n) \in \cO (n^2)$

We have:
\begin{align*}
Sum_{f}(n) &= \sum_{i=0}^{n} f(i) \\
&= \sum_{i=0}^{\lceil n_{o}\rceil - 1} f(i) + \sum_{i=\lceil n_{o}\rceil}^{n} f(i) \\
&\leq some \ constant \ d + \sum_{i=\lceil n_{o}\rceil}^{n} ci \quad (Since \ i \geq \lceil n_{o}\rceil \IMP f(i) \leq ci) \\
&= d + cn(n - \lceil n_{o}\rceil + 1) \in \cO (n^2)
\end{align*}

Therefore, we have proven that for all $f: \mathbb{N} \rightarrow \mathbb{R}^{\geq 0}$,

if  $f \in \cO (n)$, then $Sum_{f}(n) \in \cO (n^2)$ \\

$\hfill\square$ 

\newpage

\subsection*{b)}
\vspace{20pt}
Proof by induction:

We want to show that $\forall$n $\in$ $\mathbb{N}$, $\sum_{i=1}^{2^n} \frac{1}{i} \geq \frac{n}{2}$

Define the predicate $P(n)$ as $\sum_{i=1}^{2^n} \frac{1}{i} \geq \frac{n}{2}$,

where $n$ is a natural number.

We want to prove for all $n \in \mathbb{N}$ that $P(n)$ holds.\\
~\\
Base case:

$n=0$. We want to prove that $P(0)$ is true.

Since $\sum_{i=1}^{2^0} \frac{1}{i} = \sum_{i=1}^{1} \frac{1}{i} = 1 \geq \frac{0}{2} = 0, \ then \ P(n)$ holds.\\
~\\
Induction Step:

Let $k \in \mathbb{N}$, and assume that $P(k)$ is true.

That is, we assume that $\sum_{i=1}^{2^k} \frac{1}{i} \geq \frac{k}{2}$

We want to show that $P(k+1)$ holds, that is $\sum_{i=1}^{2^{k+1}} \frac{1}{i} \geq \frac{k+1}{2}$

We have:
\begin{align*}
\sum_{i=1}^{2^{k+1}} \frac{1}{i} &= \sum_{i=1}^{2^k} \frac{1}{i} + \sum_{i=2^k+1}^{2^{k+1}} \frac{1}{i}\\
&\geq \frac{k}{2} + \sum_{i=2^k+1}^{2^{k+1}} \frac{1}{i}\\
&\geq \frac{k}{2} + \sum_{i=2^k+1}^{2^{k+1}} \frac{1}{2^{k+1}}\\
&= \frac{k}{2} + (2 \cdot 2^k - (2^k+1) +1) \cdot \frac{1}{2^{k+1}}\\
&= \frac{k}{2} + 2^k \cdot \frac{1}{2^{k+1}}\\
&= \frac{k}{2} + \frac{1}{2} = \frac{k+1}{2}
\end{align*}
Thus, $\sum_{i=1}^{2^{k+1}} \frac{1}{i} \geq \frac{k+1}{2}$\\
~\\
Therefore, we have proven that $\forall n \in \mathbb{N}, \  \sum_{i=1}^{2^n} \frac{1}{i} \geq \frac{n}{2}$



$\hfill\square$ 



\newpage

\vspace{20pt}
\subsection*{c)}
Negation of the original statement:

$\exists$ f, g: $\mathbb{N} \rightarrow \mathbb{R}^{\geq 0}$, $f(n) \in \cO (g(n))$ $\wedge$ $Sum_{f}(n) \notin \cO (n\cdot g(n))$ 

$\equiv$  $\exists$ f, g: $\mathbb{N} \rightarrow \mathbb{R}^{\geq 0}$, ($\exists c_{1}, n_{o} \in \mathbb{R}^{+}, \forall n \in \mathbb{N}, n \geq n_{o} \IMP f(n) \leq c_{1}g(n)$)

$\wedge$ ($\forall c_{2}, n_{1} \in \mathbb{R}^{+}, \exists n \in \mathbb{N}, n \geq n_{1} \wedge Sum_{f}(n)\   \textgreater  \ c_{2}ng(n)$)

We want to prove its negation to disprove it. \\ 
Proof:

Let f(n) = $\frac{1}{n + 1}$, and g(n) = $\frac{1}{n}$

(1) We want to show that $\exists c_{1}, n_{o} \in \mathbb{R}^{+}, \forall n \in \mathbb{N}, n \geq n_{o} \IMP f(n) \leq c_{1}g(n)$

Let $c_{1}$ = 1 $\in$ $\mathbb{R}^{+}$ and $n_{0}$ = 1 $\in$ $\mathbb{R}^{+}$

Let n $\in \mathbb{N}$, and assume that n $\geq n_{0}$, then n $\geq$ 1

When n $\geq$ 1, f(n) = $\frac{1}{n + 1}$ $\leq$ $c_{1}$g(n) = $\frac{1}{n}$

Thus, $f(n) \in \cO (g(n))$

(2) We want to show that $\forall c_{2}, n_{1} \in \mathbb{R}^{+}, \exists n \in \mathbb{N}, n \geq n_{1} \wedge Sum_{f}(n)\   \textgreater  \ c_{2}ng(n)$

Let $c_{2}$, $n_{1}$ $\in$ $\mathbb{R}^{+}$

Let n = max\{ $n_{1}$, $2^{2c_{2}} - 1$ \}+ 1 $\in$ $\mathbb{N}$

Then, we have:
\begin{align*}
Sum_{f}(n) &= \sum_{i=0}^{n} \frac{1}{i+1} \\
&= \sum_{i=1}^{n+1} \frac{1}{i} \\
&= \sum_{i=1}^{2^{\log_{2}{(n+1)}}} \frac{1}{i} \\
&\geq \frac{\log_{2}{(n+1)}}{2} \quad (By \ \forall n \in \mathbb{N}, \  \sum_{i=1}^{2^n} \frac{1}{i} \geq \frac{n}{2} \ in  \ question \ 3b) \\
&\ \textgreater \frac{\log_{2}{(2^{2c_{2}}- 1+1)}}{2} \quad (Since \ n\ \textgreater \ 2^{2c_{2}} - 1) \\
&= \frac{2c_{2}}{2} \\
&= c_{2} \\
&= c_{2} \cdot n \cdot \frac{1}{n} \\
&= c_{2}ng(n)
\end{align*}

Thus, $Sum_{f}(n) \notin \cO (n\cdot g(n))$ \\
~\\
Therefore, by (1) and (2), we have proven that:

$\exists$ f, g: $\mathbb{N} \rightarrow \mathbb{R}^{\geq 0}$, $f(n) \in \cO (g(n))$ $\wedge$ $Sum_{f}(n) \notin \cO (n\cdot g(n))$ \\
Hence, we have disproven the original statement. \\

$\hfill\square$ 

\vspace{20pt}



\end{document}
